

% Copyright 2004 by Till Tantau <tantau@users.sourceforge.net>.
%
% In principle, this file can be redistributed and/or modified under
% the terms of the GNU Public License, version 2.
%
% However, this file is supposed to be a template to be modified
% for your own needs. For this reason, if you use this file as a
% template and not specifically distribute it as part of a another
% package/program, I grant the extra permission to freely copy and
% modify this file as you see fit and even to delete this copyright
% notice. 


\documentclass[french]{beamer}
\usepackage{subfigure}
\usepackage[utf8]{inputenc}
\usepackage[T1]{fontenc}
\usepackage{listings}
\beamertemplatenavigationsymbolsempty
\setbeamertemplate{footline}[frame number]
\setbeamerfont{footline}{size=\fontsize{14}{11}\selectfont}
% There are many different themes available for Beamer. A comprehensive
% list with examples is given here:
% http://deic.uab.es/~iblanes/beamer_gallery/index_by_theme.html
% You can uncomment the themes below if you would like to use a different
% one:
%\usetheme{AnnArbor}
%\usetheme{Antibes}
%\usetheme{Bergen}
%\usetheme{Berkeley}
%\usetheme{Berlin}
%\usetheme{Boadilla}
%\usetheme{boxes}
%\usetheme{CambridgeUS}
%\usetheme{Copenhagen}
%\usetheme{Darmstadt}
\usetheme{default}
%\usetheme{Frankfurt}
%\usetheme{Goettingen}
%\usetheme{Hannover}
%\usetheme{Ilmenau}
%\usetheme{JuanLesPins}
%\usetheme{Luebeck}
%\usetheme{Madrid}
%\usetheme{Malmoe}
%\usetheme{Marburg}
%\usetheme{Montpellier}
%\usetheme{PaloAlto}
%\usetheme{Pittsburgh}
%\usetheme{Rochester}
%\usetheme{Singapore}
%\usetheme{Szeged}
%\usetheme{Warsaw}

\title{Génération procédurale de planètes}

\author{  \vspace{1.5cm}Rémy \textsc{Maugey} \and
  Jérémi \textsc{Bernard} \and
  Hugo \textsc{Alonso} \and
  Brian \textsc{Mazé}\\
  \vspace{2.5cm}
  Client : Emmanuel \textsc{Fleury}}

\date{11 avril 2018}

\AtBeginSection[]{
  \begin{frame}
  \vfill
  \centering
  \usebeamercolor[fg]{title}
  \usebeamerfont{title}\insertsectionhead\par
  \vfill
  \end{frame}
}

\begin{document}

\begin{frame}
  \titlepage
\end{frame}

\begin{frame}{Présentation du sujet}

  
    \begin{itemize}
  \item
    Génération procédurale de planète
    \begin{itemize}
    \item
      Utilisation de la méthode \alert{"Continuous Distance-Dependent Level of Detail for Rendering Heightmaps"} (CDLOD) présenté par Filip Strugar en Juillet 2010\protect\footnotemark
    \end{itemize}
  \end{itemize}
  
  \begin{figure}
   \includegraphics[scale=0.20]{img/planet_intro.png}
   \caption{Planète}
\end{figure}
  
      \footnotetext{Dernier accès avril 2018: \url{http://www.vertexasylum.com/downloads/cdlod/cdlod_latest.pdf}}
    
\end{frame}

\section{CDLOD}

\begin{frame}{CDLOD}{Carte de Hauteur}


\begin{itemize}
    \item Carte de hauteur (Heightmap)
    \begin{itemize}
        \item Image en nuance de gris. 
        \item Le noir représente la distance minimum et le blanc la distance maximale.
        \item Valeur interprétée comme une élévation par rapport à la surface.
    \end{itemize}

\end{itemize}


\begin{figure}
   \includegraphics[scale=0.60]{img/heightmap.png}
   \caption{Interprétation de la carte de hauteur \protect\footnotemark}
\end{figure}
    \footnotetext{Source, dernier accès avril 2018: \url{https://newbiz.developpez.com/tutoriels/opengl/heightmap/}}

\end{frame}

\begin{frame}{CDLOD}{Quadtree}


\begin{itemize}

    \item Quadtree
    \begin{itemize}
        \item Structure de données
        \item Arbre où chaque n\oe{}ud possède quatre fils
        \item Un bon moyen de diviser un terrain en zone égale à différents niveaux
        
        
    \end{itemize}
\end{itemize}

\begin{figure}
   \includegraphics[scale=0.40]{img/quadtree-arbre.png}
   \caption{Quadtree \protect\footnotemark}
\end{figure}
    \footnotetext{Source, dernier accès avril 2018: \url{http://codeforces.com/blog/entry/57498}}
    
\end{frame}

\begin{frame}{CDLOD}{Sélection}

\begin{itemize}
    \item Amélioration des performances en réduisant le nombre de triangles utilisés
    \begin{itemize}
        \item Sélection de n\oe{}ud dans le quadtree
        \begin{itemize}
            \item Distance entre la camèra et le terrain
            \item Champ de vision
        \end{itemize}
    \end{itemize}
\end{itemize}


\begin{figure}
   \includegraphics[scale=0.55]{img/selection.png}
   \caption{N\oe{}uds affichés \protect\footnotemark}
\end{figure}
    \footnotetext{Source, dernier accès avril 2018: \url{http://www.vertexasylum.com/downloads/cdlod/cdlod_latest.pdf}}
    

\end{frame}


\begin{frame}{CDLOD}{Discontinuité / Trou}
    

    \begin{itemize}
    \item Discontinuité 
    \begin{itemize}
        \item La différence entre deux niveaux de détails voisins va faire apparaître des discontinuités
        \item Discontinuité présente par exemple dans la méthode "Chuncked-LOD" présenté par Thatcher Ulrich en 2002\protect\footnotemark
        \footnotetext{Dernier accès avril 2018: \url{http://tulrich.com/geekstuff/sig-notes.pdf}}
    \end{itemize}
\end{itemize}

\begin{figure}
   \includegraphics[scale=0.7]{img/cracks.png}
   \caption{Discontinuité \protect\footnotemark}
\end{figure}
    \footnotetext{Source, dernier accès avril 2018: \url{https://people.eecs.berkeley.edu/~sequin/CS284/LECT09/L13.html}}
    

\end{frame}

\begin{frame}{CDLOD}{Morphing}
    
\begin{itemize}
    \item Morphing
    \begin{itemize}
        \item Transition fluide
        \item Opération effectué avant le changement de niveau de détail.
    \end{itemize}
\end{itemize}

\begin{figure}[H]
\centerline{
   \subfigure[]{\includegraphics[scale=0.44]{img/morph5.png}}
   \subfigure[]{\includegraphics[scale=0.44]{img/morph4.png}}
   \subfigure[]{\includegraphics[scale=0.48]{img/morph3.png}}
   }
\end{figure}
\begin{figure}[H]
   \subfigure[]{\includegraphics[scale=0.63]{img/morph2.png}}
   \subfigure[]{\includegraphics[scale=0.8]{img/morph1.png}}
   
   \caption{Augmentation du niveau de détail}
\end{figure}


\end{frame}

% -------------
% Besoins

\begin{frame}{Besoins}
  Besoins fonctionnels:
  \begin{itemize}
    \item Algorithme de CDLOD
      \begin{itemize}
        \item Quadtree
        \item Morph
      \end{itemize}
    \item Interface graphique
    \item Génération procédurale
      \begin{itemize}
        \item Carte de hauteur
        \item Différents bruits
      \end{itemize}
  \end{itemize}
  Besoins non-fonctionnels:
  \begin{itemize}
    \item CMake comme système de configuration
    \item Utiliser le standard C++11 ou C++14
    \item Cibler Linux avec des drivers libres (OpenGL 3.3)
    \item Implémenter des tests
  \end{itemize}
\end{frame}




%%  ARCHI

  	
  %-----------------------------------------------------------------%
  %   Architecture et explication détaillé de l'implémentation
  %-----------------------------------------------------------------%
  \chapter{Architecture du programme}
  
  \section{Lancement du programme}
  A été rajouté au programme des options de lancement, l'usage est disponible 
  
  \section{Carte de hauteur}
  
  Le programme utilise un \emph{quadtree} pour générer les différents niveaux de détail. 
  Ensuite une nouvelle hauteur est appliquée à ce point pour le déplacer. 
  Cependant, le \emph{quadtree} ne stocke pas la hauteur du sommet. Cette dernière
  est enregistrée dans une texture 2d et est envoyé à la carte graphique. 
  Chaque pixel de la texture représente donc une valeur codée sur un nombre flottant entre -1 et 1.
  La texture récupérée par la carte graphique, est plaquée sur le maillage de la sphère et la hauteur du sommet
  est déduite de l'interpolation des texels de la texture sur le maillage.
  %définir texel
  %définir interpolation ?
  
  
  \section{Système de rendu}
  Le programme est découpé en classes afin d'abstraire les appels à la bibliothèque OpenGL. La figure \ref{fig:uml_scene} représente le diagramme des classes qui forment l'abstraction principale du programme.
  Les classes relatives à la génération, l'affichage de la planète et l'utilisation du CDLOD sont strockés dans le dossier
  PlanetTech.
  
  \begin{figure}
  \centering
  \includegraphics[width=10cm]{img/uml_scene.png}
  \caption{Diagramme de classe de Planet}
  \label{fig:uml_scene}
  \end{figure}

  La classe Scene est la classe centrale du programme, elle permet
  d'initialiser et de créer la planète. Scene gère les paramètres
  principaux du programme à travers les entrées clavier d'InputManager.
  Ces paramètres sont le nombre de niveaux de détail, le mode d'affichage
  (texture + maillage, maillage uniquement, texture uniquement). Scene
  gère aussi l'affichage des informations comme le compteur d'images par
  seconde ou le nombre de sommets actuellement affiché.
  
  La classe Planet représente la sphère à afficher. Ses classes filles
  spécialisent le type de planète à afficher, ce sont elles qui
  définissent la taille de la sphère, sa texture et sa carte de hauteur.\\
  \textbf{graphe d'héritage de Planet quand Procedural sera merge}
  
  %\begin{figure}
  %centering
  %\includegraphics[width=10cm]{img/plan_exec.png}
  %\caption{Plan}
  %\label{fig:triangulator}
  %\end{figure}
 
  La figure \ref{fig:plan} représente le graphe simplifié de
  l'exécution du programme. L'étape d'initialisation se fait une seule
  fois au début du programme, et les étapes de mise à jour et d'affichage
  se répètent tout le long de l'exécution.
  
  \section{Triangulator}
  La classe Triangulator permet de générer le \emph{quadtree} utilisé pour le système de niveau de détail.
  Un objet Frustum est utilisé pour décider si un triangle de l'arbre est ou non dans le cône 
  de vision de la caméra. Le diagramme de classes présenté en figure \ref{fig:plan} présente la classe
  Triangulator. Les différents composants seront analysés plus loin dans le chapitre implémentation.
  
  \section{Patch}
  La classe Patch permet de générer et d'afficher le maillage en y appliquant l'algorithme de morphing.
  Le morphing est ce qui permet d'avoir des transissions douces entre les différents niveaux de détails.
  Les détails seront expliqués plus loin dans la partie implémentation.
  
  
  \begin{figure}
  \centering
  \includegraphics[width=10cm]{img/triangulator.png}
  \caption{Diagramme de classe de Triangulator}
  \label{fig:plan}
  \end{figure}
  
  \section{Frustum}
  La classe Frustum permet de faire des opérations sur les triangles qui sont dans le cône de vision de la caméra. Le Frustum utilisé par la classe Triangulator pour permettre les tests d'union entre les triangles est le cône de vision. La classe Frustum permet aussi de décider si un triangle est à afficher ou non.\\
  
  Cette classe permet entre autres d'éliminer les triangles qui ne sont pas visibles de la caméra. Cette opération appelée \textit{culling} permet d'éviter d'envoyer des triangles inutiles à la carte graphique.
  Dans ce cas ci,-cela permet aussi d'éliminer ces triangles de l'arbre et donc d'optimiser sa génération.
  
 \newpage %pour placer les images correctement
 

%%

\section{Triangulator et Patch}
\begin{frame}{Triangulator}{Icosaèdre}
  Le triangulator construit un maillage de triangle, en subdivisant récursivement les faces d'un icosaèdre. Le but étant d'approcher une sphère.
%parler du fait que le triangulator considère uniquement les coordonée par rapport au centre de la planète, et travaille dans l'espace en 3 dimensions, sur des triangles parfait (pas de relief)
\begin{figure}[H]
\centerline{
   \subfigure[Icosaèdre de base]{\includegraphics[scale=0.1]{img/3-Icosahedron.jpg}}
   \subfigure[Récursion total de profondeur 1]{\includegraphics[scale=0.1]{img/4-IcosphereSd1.jpg}}
   \subfigure[Récursion total de profondeur 2]{\includegraphics[scale=0.1]{img/5-IcosphereSd2.jpg}}
   \subfigure[Récursion total de profondeur 3]{\includegraphics[scale=0.1]{img/6-IcosphereSd3.jpg}}
   }
\end{figure}
    \footnotetext{Source, dernier accès avril 2018: \url{http://robert-lindner.com/img/blog/planet_renderer/week5-6/researchPaper.pdf}}
\end{frame}

\begin{frame}{Triangulator}{Subdivision}
\begin{figure}
%todo gérer borders
\centerline{
   \subfigure[]{\includegraphics[scale=0.25]{img/triangleBeforeSplit.png}}
   \subfigure[]{\includegraphics[scale=0.25]{img/split1.png}}
   \subfigure[]{\includegraphics[scale=0.25]{img/splitBombe.png}}
   }
   \caption{Subdivision d'un triangle, Les enfants sont obtenues en prenant le milieu de chaque coté du père, puis en les normant sur la sphère}
   
\end{figure}
\end{frame}

\begin{frame}{Triangulator}{Récursion}
Chaque appel récursif traite un triangle, et s'appuie sur la fonction SplitHeuristic, qui décide si le triangle doit ètre:
\newline
  \begin{itemize}
  \item {
    \textbf{filtré}
  }
  \item {
    \textbf{validé}
  }
  \item {
   ou \textbf{subdivisé}
  }
  \end{itemize}
\end{frame}

\begin{frame}{Triangulator}{Filtre}
Pour chaque appel, la récursion est arrêtée d'office si le triangle est totalement hors du frustum, ou si il fait face à la mauvaise direction.
\begin{figure}
%todo gérer borders
\centerline{
   \subfigure[Frustum culling]{\includegraphics[scale=0.15]{img/culling.png}}
   \subfigure[Backface culling]{\includegraphics[scale=0.094]{img/backfaceCulling.jpg}}
   }
   
\end{figure}
      \footnotetext{Source dernier accès avril 2018: \url{http://robert-lindner.com/img/blog/planet_renderer/week5-6/researchPaper.pdf}}
  
  \end{frame}
  
 \begin{frame}{Triangulator}{Validation}
     

Si le triangle passe les filtres, alors soit :
\begin{itemize}
    \item Il est suffisamment proche pour être éligibles par la table des distances et est subdivisé en 4 sous triangles.
    %explication table des distance
    \item Soit il est envoyé au GPU.
\end{itemize}

La table des distances est un simple tableaux indiquant quels sont les limites pour chacun des niveaux.
\newline

[14000,7000,3500,1750,875...]
\newline

Un triangle de niveaux 2 doit être éloigné de moins de 7000 kilomètres pour être subdivisé. 
\label{fig:my_label}

 \end{frame}
 
 \begin{frame}{Patch}{Découpage}
 Patch permet d'organiser en amont le redécoupage et le morphing des triangles envoyés par Triangulator dans le GPU.
 \begin{figure}
\caption{Découpage d'un triangle dans la carte graphique. Tous les points sont cette fois ci normés selon leurs position dans la heightmap}

\centerline{
   \subfigure[Triangle sortant de la récursion]{\includegraphics[scale=0.15]{img/patchBefor.png}}
   \subfigure[Triangle "patché"]{\includegraphics[scale=0.15]{img/patch.png}}
   \subfigure[Triangle après normalisation selon la carte de hauteur ]{\includegraphics[scale=0.15]{img/afterPatchNorm.png}}
   }
   
   
\end{figure}
 \end{frame}

 


% -----
% TESTS
% -----
% TODO
% Thème pour listings
% Transitions entre les diapos


\section{Tests et analyse du code}
% Tests unitaires pour assurer la non régression
% Tests de performances pour mesurer les capacités du programme

% ---

% Remplacer par diapo de titre ?
  %\begin{frame}{Tests}{Objectifs}
  %  \begin{itemize}
  %  \item {
  %    S'assurer du bon fonctionnement de l'application reprise
  %  }
  %  \item {
  %    Vérfier que les modifications apportées au projet sont
  %    fonctionnelles
  %  }
  %  \end{itemize}
  %\end{frame}

% ---

  \begin{frame}{Tests}{Mise en place}
    Intégration dans le système de compilation:
    \begin{itemize}
      \item Un exécutable par test, compilé avec le reste du programme
        % Permet de facilement tester une seule fonctionnalité
      \item Lancement à partir du système de compilation (module CTest
        de CMake)
        % Facilité d'utilisation pour le développeur et pour un logiciel
        % d'Intégration Continue afin de lancer les tests
        % automatiquement. CTest permet de récolter des statistiques sur
        % les tests (programes de dashboard)
    \end{itemize}
  \end{frame}

% ---
  
  \begin{frame}[fragile]{Tests}{Fonctionnement}
    \textbf{Mocking}: remplacer certaines parties du programme par des
    faux afin de faciliter le test d'une fonctionnalité.
    % Dans note projet:
    \begin{itemize}
      \item Remplacer la définition d'une classe ou d'une fonction par
        un faux pour contrôler son comportement
        % Utilisé dans le projet pour "désactiver" certaines classes
        % dans le but de simplifier la construction des tests. Par
        % exemple la classe Shader inutile aux tests, ou désactiver les
        % appels OpenGL car pas de contexte
      \item Rediriger un appel de fonction pour l'effacer ou récupérer
        ses arguments grâce au préprocesseur
        \begin{lstlisting}[language=C++, basicstyle=\small]
#define glActiveTexture         (void)sizeof
// ...
#define glBufferData(t,size,buff,m) test(buff, size)
        \end{lstlisting}
    \end{itemize}
  \end{frame}

% ---

  %\begin{frame}[fragile]{Tests}{Mocking}
  %  \begin{itemize}
  %    \item Effacer un appel:
  %      \begin{lstlisting}[language=c++]
  %// Sans valeur de retour
  %#define glActiveTexture         (void)sizeof
  %// Avec valeur de retour
  %#define glCreateShader          0;(void)sizeof
  %      \end{lstlisting}
  %    \item Redirection:
  %      % Avec masquage de certains paramètres.
  %      \begin{lstlisting}[language=c++]
  %void test_update_buffer(void *buffer, size_t size);
  %// ...
  %#define glBufferData(type, size, buffer, mode) test_update_buffer(buffer, size)
  %      \end{lstlisting}
  %  \end{itemize}
  %\end{frame}

% ---

  \begin{frame}{Tests}{Exemple de test unitaire}
    \texttt{Triangulator::SplitHeuristic}:
    % Test en boite blanche: partie de l'implémentation désactivée
    % (culling) pour tester la décision de couper ou non
    % Repose sur la connaissance de l'implémentation du test. Les
    % scénarios créés sont conçus en connaissant à l'avance le résultat
    % attendu.
    \begin{itemize}
      \item Entrées: un triangle du maillage
      \item Sorties: Le status du triangle:
        \begin{itemize}
          \item Masqué
          \item A découper
          \item Feuille
        \end{itemize}
    \end{itemize}
  \end{frame}

% --- 

  \begin{frame}{Tests}{Tests de performance}
    Tests de performance de la génération de la carte de hauteur pour
    des résolutions allant de $2^6\times2^6 (64\times64)$ à
    $2^{11}\times2^{11} (2048\times2048)$
    \begin{figure}[!ht]
      \centering
      \includegraphics[width=10cm]{img/bench_i53570_fixed.png}
      \caption{Temps de génération en fonction de la complexité de la
      planète (Intel Core I5 3570)}
      \label{fig:bench}
    \end{figure}
    % Valeurs numériques données par le programme de test et importées
    % dans un tableur
  \end{frame}

% ---

  \begin{frame}{Tests}{Analyse statique}
    Objectifs:
    \begin{itemize}
      \item Améliorer la lisibilité:
        Utiliser \texttt{auto} pour ne pas répéter le type
        % Et donc la maintenabilité du code pour éviter les erreurs
      \item Moderniser le code:
        Utiliser \texttt{nullptr} au lieu de \texttt{NULL} ou de
        \texttt{0}
        % Exploiter au maximum les avantages du standard C++ utilisé
      \item Améliorer les performances
        Utiliser \texttt{emplace\_back} dans les conteneurs de la STL au
        lieu de \texttt{push\_back}.
        % Un seul constructeur au lieu d'une constructeur + copie ou
        % déplacement
    \end{itemize}
    Analyseurs utilisés:
    \begin{itemize}
      \item Clang / GCC (\texttt{-Wall -pedantic})
      \item clang-tidy
      \item cppcheck
    \end{itemize}
    Avantages:
    \begin{itemize}
      \item Intégration dans CMake
      \item Facilement configurables
      \item Correcteur automatique de clang-tidy
    \end{itemize}
  \end{frame}

% ---

\begin{frame}[fragile]{Bugs corrigés}{}
\begin{itemize}
    \item 
    Il y avait un dépassement de tableau dans le GPU(Shaders/patch.glsl l 35) lorsque la récursion s'arrêtait à l'icosaèdre (ie: lorsque la planète était trés éloignée):
\begin{lstlisting}[language=c++]
float low = distanceLUT[lev-1];
\end{lstlisting}
    \item
    La fonction SplitHeuristic désactive la vérification du frustum culling pour les futurs niveaux lorsque un triangle est totalement inclue. La fonction alternait anormalement entre ce mode "frustum culling", et le mode subdivision direct, engendrant des vérifications inutiles.
    
\end{itemize}


        
  \end{frame}
% ---





\begin{frame}{Conclusion}

Ajouts

\begin{itemize}
    \item Portage sous \textit{CMake} (GENie)
    \item Analyse statique  (-Werror ...)
    \item Analyse dynamique (valgrind, callgrind ...)
    \begin{itemize}
        \item Modernisation du code
        \item Correction de bugs
    \end{itemize}
    \item Génération des cartes de hauteurs
    \item Gestion des paramètres d'entrées
    \item Tests
\end{itemize}

\end{frame}

%---------
%ANNEXE
%--------

  \chapter*{Annexe}
  \addcontentsline{toc}{chapter}{Annexe}


  \section*{Chuncked LOD}
  \label{sec:chunked-lod}

  
  la méthode \textit{CLOD} "Chuncked LOD" développé par Thatcher Ulrich~\cite{CLOD}, sur laquelle est basé la méthode \textit{CDLOD}.

\begin{figure}[!ht]
\centerline{
    \includegraphics[width=11cm,height=3.5cm]{img/clod.png}}
    \caption[CLOD]{CLOD\protect\footnotemark}
    \label{fig:clod}
\end{figure}
\footnotetext{Extrait de \url{http://tulrich.com/geekstuff/sig-notes.pdf}, dernier accès Mars 2018}


  La méthode \emph{CLOD} utilise un \textit{quadtree} permettant de stocker les différents \emph{LOD}. Lors de l'exécution le \emph{quadtree} est généré et composé des niveaux de détails, les \emph{LOD} nécessaire sont calculés et ainsi utilisés depuis le \emph{quadtree}. 
  
  Pour appliquer une texture il suffit alors d'associer chaque "chunck" (n\oe{}ud) à une texture.
  Le rendue du terrain est alors fait en fonction de la distance par rapport à la caméra. Pour affiché un n\oe{}ud, le \emph{quadtree} est parcourue depuis la racine avec une tolérance maximum prédéfinie. Si le du n\oe{}ud parcourue est valide il est affiché et si il y a un trop grand écart alors on vérifie ses fils dans \emph{quadtree}, et ainsi de suite.
 
  Lorsqu'un n\oe{}ud est sur le point d'être réduit à ses fils ou inversement, il va alors se produire un effet "flash"/une apparition brusque de la grille. La méthode résout se problème en ajoutant un petit \emph{morphing} (une transition fluide). Pour se faire la méthode utilisé consiste à ajouter un point et de le décaler pour que cela corresponde niveau de détail souhaité(plus grand ou plus petit)
 
 
 \begin{figure}[!ht]
 \centerline{
    \includegraphics[width=8cm,height=2cm]{img/morph-pop.png}}
    \caption[morph]{ Illustration \protect\footnotemark}
    \label{fig:morph-pop}
\end{figure}
\footnotetext{Extrait de \url{http://tulrich.com/geekstuff/sig-notes.pdf}, dernier accès Mars 2018}

\newpage
  Dans l'exemple de la figure \ref{fig:morph-pop} la ligne (1) est composé de deux sommets (A et B),on cherche ensuite à augmenter le niveau de détail de cette ligne en y ajoutant un sommet. On va donc placer un troisième sommet (C) au milieu de la ligne composant donc une ligne plus détaillé (2). Le troisième sommet ainsi placé va ensuite être déplacé progressivement pour atteindre son propre emplacement, et représenter le "bon détail" (3).\\

 Un nouveau problème va donc par la suite apparaître. Comme montré et expliqué au début du chapite \ref{chap:cdlod}, la différence entre deux niveaux de détails voisins va faire apparaître des discontinuité. Ulritch.T pour se faire va alors en quelque sorte "remplir" ces trous en créant un "rideau" composé de la même texture dont il va se servir pour combler le vide entre les n\oe{}uds.
 Le premier problème de cette méthode est que pour s'assurer que le "rideau" couvre la totalité de la zone vide, ce dernier doit partir de la plus grande hauteur de la zone et descendre à la vertical en dessous de la totalité du niveau de détail. C'est à dire peut-être(et même dans la plus part du temps) au dessous du niveau actuel. Cela va donc crée une zone affiché mais non visible.

 \begin{figure}[!ht]
    \includegraphics[width=12cm]{img/skirt.png}
    \caption[morph]{ Résolution du problème de discontinuité \protect\footnotemark}
    \label{fig:skirt}
\end{figure}
\footnotetext{Extrait de \url{http://tulrich.com/geekstuff/sig-notes.pdf}, dernier accès Mars 2018}

Dans l'exemple de la figure \ref{fig:skirt}, sur la figure de gauche, une discontinuité est apparue car la zone de gauche est à un niveau plus haut que le point formé par la zone de droite, qui après avoir glisser(comme vue dans l'exemple de la figure \ref{fig:morph-pop}) va se situer plus bas est crée un vide.
Le rideau va ainsi se créer (visible sur la figure de droite). On peut ainsi bien observer que le rideau va au delà de la discontinuité.

Le deuxième problème de cette méthode, est que la texture appliqué à se rideau est la même que l'arête où se produit la discontinuité. Il n'y aura donc en effet pas de zone vide, mais la texture en elle même ne serra pas propre au triangle.
  
 \section*{Usage}

\begin{center}
\lstset{language=sh}
\begin{lstlisting}[basicstyle=\small]

./PlanetRenderer <NOISE> --<noise-option>=<value> --<option>

option:
	 -h  --help         show this page.
 	     --fullscreen   start program in fullscreen.
 	
<NOISE> --<noise-option>=<value>
Order of option does not matter.

Common options                     Default value
	 --width=Int                     800
	 --height=Int                    800
	 --max_height=Float              10.0

SIMPLEX
	<common-options>

RIDGEG-NOISE
	<common-options>

FLOW-NOISE
	<common-options>
	--angle=Float                    0.5

FBM
	<common-options> 
	--octave=UInt                    4
	--lacunarity=Float               2.0 
	--gain=Float                     0.5

WARPED-FBM 
	<common-options> 
	--octave=UInt                    4
	--lacunarity=Float               2.0
	--gain=Float                     0.5

DFBM-WARPED-FBM 
	--octave=UInt                    4
	--lacunarity=Float               2.0
	--gain=Float                     0.5

RIDGED-MULTI-FRACTAL 
	<common-options> 
	--octave=UInt                    4
	--lacunarity=Float               2.0
	--gain=Float                     0.5

\end{lstlisting}
\end{center}


\end{document}