\chapter{Tests}\label{tests}

Tests effectués sur le projet: tests unitaires sur la partie génération
de la géométrie de la planète ainsi qu'analyse du code existant à l'aide
d'outils externes.

\section{Mise en place}\label{mise-en-place}

Les tests sont intégrés dans notre système de compilation. Ils sont
compilés avec le projet mais lancés séparément.

Pour certains tests, nous avons eu besoin de désactiver certaines
parties du projet pour réduire le nombre de dépendances de chaque test.
En effet certaines classes à tester dépendait d'une autre, qui dépendait
d'une autre etc. Lorsque les dépendances ne faisait pas partie des
tests, nous les avons remplacé par de fausses classes vides. Cela permet
de réduire le temps de compilation des tests et réduit la taille des
exécutables.

Il a aussi été nécessaire de désactiver les appels à OpenGL dans les
classes testées car il n'y a pas de contexte graphique. Pour cela nous
avons utilisé la méthode du ``mocking'', c'est à dire remplacer certains
appels de fonction par d'autres. Pour cela, nous utilisons des macros
comme celle ci:

\lstset{language=C++}
\begin{lstlisting}
#undef  glActiveTexture
#define glActiveTexture         (void)sizeof
\end{lstlisting}

Remplacer l'appel de fonction par un appel à \texttt{sizeof} permet
d'effacer les paramètres passés à la fonction, en calculant leur taille.
Le \texttt{cast} en type \texttt{void} spécifie au compilateur que la valeur
n'est pas utilisée. Si la fonction avait une valeur de retour, une
valeur de remplissage peut être placée avant le \texttt{sizeof} comme
ceci:

\lstset{language=C++}
\begin{lstlisting}
#define glCreateShader          0;(void)sizeof
\end{lstlisting}


\section{Tests}\label{tests-1}

\subsection{Triangulator}\label{triangulator}

\subsubsection{SplitHeuristic}\label{splitheuristic}

La fonction SplitHeuristic est à la base de la génération du maillage de
la planète. Elle sert à déterminer si un triangle est visible ou non
(pyramide de vision de la caméra, masqué par le reste de la planète), et
s'il doit être découpé en 4 triangles fils, cette fonction fait partie
du parcours du quadtree.

Fonctionnement:

Ce test est un test en boite grise, il dépend d'une partie de
l'implémentation de la fonction. Le test utilise de fausses
implémentations vide de certaines dépendances de \texttt{Triangulator},
Shader et Frustum qui ne sont pas utilisées dans le test. Le culling des
triangles est désactivé dans ce test afin de simplifier sa mise en
place. Le test crée deux cas à tester, un triangle qui devrait être
coupé en quatre, et un triangle qui ne devrait pas être coupé. La
fonction est appelée une fois sur chaque triangle et sa valeur de retour
étudiée.

La mise en place du test a permis de se rendre compte d'un bug dans
l'implémentation de la fonction. \textbf{Explications Hugo sur
split/splitcull}.

\subsubsection{RecursiveTriangle}\label{recursivetriangle}

RecursiveTriangle prend en entrée un triangle, un niveau de détail et
une option pour activer le culling par pyramide de vision de la caméra.
Elle enregistre dans un tableau de la classe l'ensemble des triangles
créés récursivement. Cette fonction parcours le quadtree.

Fonctionnement: Ce test crée deux cas, un où le triangle doit être coupé
et un où le triangle ne doit pas être coupé et vérifie que le nombre de
triangle est bien celui attendu.

\subsubsection{Triangulator::GenerateGeometry}\label{triangulatorgenerategeometry}

A écrire

\subsection{TextRenderer::UpdateBuffer}\label{textrendererupdatebuffer}

\texttt{UpdateBuffer} prend des entrées via la fonction
\texttt{DrawText}, sous la forme d'un tableau de chaînes de caractère à
écrire et de leurs positions sur l'écran. La fonction passe ses
résultats à OpenGl via la fonction \texttt{glBufferData}. Le tableau de
données envoyé à OpenGl contient des points correspondants aux coins
supérieurs gauche de chaque lettre à afficher.

Fonctionnement: Le test est mis en place via ``mocking'' de la fonction
\texttt{glBufferData}, en la redirigeant vers une fonction de test qui vérifie le
contenu du tableau pour donner le résultat du test. Le test écrit du
texte à une position connue, et vérifie que les points créés sont bien
placés.

\subsection{Patch::GenerateGeometry}\label{patchgenerategeometry}

A écrire

\newpage
\section{Analyse statique}\label{sec:sanal}

Afin d'améliorer la qualité générale du code du projet, il est
intéressant d'utiliser des outils afin d'analyser le code source dans le
but de trouver et corriger de potentielles erreurs. Ces outils peuvent
être de simplement activer des options du compilateur (augmenter les
niveaux de ``warnings'', les avertissements compilateur), ou encore des
outils externes d'analyse statique comme par exemple \texttt{clang-tidy}
ou \texttt{cppcheck}. Ces outils s'intègre dans notre système de
configuration CMake, et sont lancés pendant la compilation.

\section{Configuration}\label{configuration}

Il est souhaitable d'activer le plus d'options d'analyse possible,
cependant certains avertissements ne sont pas utiles pour améliorer
notre programme, c'est pourquoi nous avons désactivé plusieurs classes
d'avertissement dans notre configuration:

\begin{itemize}
\item
  Politique de Google sur les tailles
  d'entiers\footnote{\url{https://google.github.io/styleguide/cppguide.html\#Integer_Types}, dernier accès Mars 2018}. Google recommande de remplacer
  les types d'entiers de tailles non fixes (\texttt{short},
  \texttt{int}, etc) par des types de tailles connues
  (\texttt{int16\_t}, \texttt{int32\_t}). En effet, le standard C++
  défini \texttt{int} comme ayant une taille d'au moins 16 bits par
  exemple, alors que la plupart des développeurs présument que la taille
  du type \texttt{int} est d'au moins 32 bits. Cette règle n'existe que
  dans un soucis de sécurité afin d'éviter les comportements indéfinis
  lors de dépacements d'entier. Le problème ne se pose pas dans le cadre
  de notre projet car l'architecture cible est connue à l'avance. Dans
  un soucis de simplicité et de lisibilité, nous ne prenons pas compte
  de cet avertissement.
\item
  ``Avoid naked \texttt{union}s'' (éviter d'utiliser les \texttt{union}
  du C), recommandation de la Fondation du Standard
  C++\footnote{\url{http://isocpp.github.io/CppCoreGuidelines/CppCoreGuidelines\#Ru-naked} , dernier accès Mars 2018}. Cet avertissement est levé
  lorsqu'on accède à un membre d'une \texttt{union}. Il est fondé sur le
  faits que les \texttt{union}s n'assurent aucune sécurité quand à
  l'accès à leurs membres, c'est à dire que l'on pourrait accèder à un
  membre alors que la valeur contenue dans l'\texttt{union} ne
  correspond pas à ce type, résultant en un comportement indéfini. Les
  \texttt{union}s sans sécurités sont dangereuses, cependant elles sont
  utilisées dans la bibliothèque GLM, utilisée dans notre projet. Ces
  avertissements proviennent de la bibliothèque sur laquelle nous
  n'avons pas de contrôle, afin de ne pas surcharger les retours des
  analyseurs cet avertissement a été désactivé.
\item
  ``Keep use of pointers simple and straightforward'' (garder
  l'utilisation des pointeurs simple et
  évidente)\footnote{\url{http://isocpp.github.io/CppCoreGuidelines/CppCoreGuidelines\#es42-keep-use-of-pointers-simple-and-straightforward}, dernier accès Mars 2018}. Cet avertissement recommande de
  ne jamais utiliser de pointeurs pour représenter un tableau, et de ne
  pas faire de calculs sur les adresses des pointeurs. Les erreurs sur
  les accès en dehors de l'espace mémoire alloué à un pointeur sont des
  comportements indéfinis. Afin d'éviter tous risques, il est préférable
  de respecter cette règle. Cependant, le projet que nous avons repris
  ne respecte pas cette règle, et les analyseurs génèrent beaucoup
  d'avertissements. Afin de s'assurer qu'il n'existait pas de problèmes
  d'accès mémoire, nous avons utlisé des outils d'analyse de mémoire,
  sur lesquels nous reviendront plus tard. Les outils d'analyse de
  mémoire ne relevant aucuns problèmes, nous avons décidé comme solution
  à court terme de désactiver cet avertissement afin de simplifier
  l'étude des autres avertissements. Une solution à long terme sera de
  remplacer les tableaux bruts par les tableaux de la STL
  (\texttt{std::array} ou \texttt{std::vector}). En plus d'une interface
  plus complète, ces types proposent des sécurités de vérification
  d'accès en dehors du tableau en activant les options de débogage des
  implémentations de la STL.
\item
  ``Do not pass an array as a single pointer'' (ne pas passer un tableau
  en tant que
  pointeur)\footnote{\url{http://isocpp.github.io/CppCoreGuidelines/CppCoreGuidelines\#i13-do-not-pass-an-array-as-a-single-pointer}, dernier accès Mars 2018}. Dans le même esprit que
  l'avertissement précédent, il est déconseillé de passer un tableau en
  tant que pointeur. Cet avertissement a été désactivé pour les mêmes
  raisons.
\item
  ``Don't use \texttt{reinterpret\_cast}'' (ne pas utiliser
  \texttt{reinterpret\_cast})\footnote{\url{http://isocpp.github.io/CppCoreGuidelines/CppCoreGuidelines\#Pro-type-reinterpretcast}, dernier accès Mars 2018}. Le cast
  \texttt{reinterpret\_cast} permet de demander au compilateur
  d'interpréter une valeur d'un type comme étant d'un autre type. Cela
  permet entre autre d'utiliser la valeur d'un nombre entier en tant
  qu'une adresse à mettre dans un pointeur. Ce cast est dangereux car
  aucune vérification n'est effectuée, et les raisons de l'utiliser sont
  rares. Cependant, dans notre cas ce cast est très utile à cause d'un
  défaut de conception de la bibliothèque OpenGL. Nous reviendrons sur
  les raisons de l'utiliser plus tard. Son utilisation dans notre projet
  étant légitime et assez courante, nous avons décidé de désactiver cet
  avertissement.
\end{itemize}

\section{Correctifs}

\subsection{Lisibilité}

\subsubsection{Entourer les structures de contrôle par des
accolades}\label{entourer-les-structures-de-contruxf4le-par-des-accolades}

Cette simple amélioration permet d'éviter les erreurs d'inattention en
oubliant de rajouter les accolades après avoir rajouter une ligne au
bloc.

\subsubsection{Conversions de type}\label{conversions-de-type}

Le C++ introduit un ensemble de casts nommés pour remplacer le cast
universel du C. L'objectif et d'exprimer l'intention du cast clairement
dans le code, le rendant plus lisible et moins susceptible aux erreurs.
Les casts les plus simples ont été remplacés par \texttt{clang-tidy}
automatiquement par des \texttt{static\_cast}, effectuant une conversion
de la valeur vers un autre type.

Cependant, certaines conversions un peu plus complexes ont lieu dans le
code. En effet, certaines fonctions de la bibliothèque OpenGL attendent
un nombre en paramètre, mais le paramètre en question est de type
\texttt{GLvoid\ *}, c'est à dire un pointeur. Les conversions implicites
d'un entier vers un pointeur étant illégales en C++, un cast est
nécessaire dans ces conditions.  L'utilisation des casts universels du C
étant déconseillée, le cast C++ \texttt{reinterpret\_cast} permet de le
remplacer en demandant au compilateur d'interpréter la variable comme
étant d'un type différent, sans effectuer de conversion de la valeur.

