\chapter*{État de l'art}
\addcontentsline{toc}{chapter}{État de l'art}
\setcounter{chapter}{0}

\textbf{Inchangé}

En vue de l'ampleur du projet, il est préférable de reprendre un projet
déjà existant et de l'étendre pour compléter nos besoins. Nous devons
cependant respecter les contraintes du client. Notamment, le projet doit
être écrit en C++ et visant Linux. Nous reviendrons plus en
détail sur ces contraintes, mais le système d'exploitation visé réduit
nos choix pour reprendre un projet existant. Dans un premier temps, nous
nous intéresserons à la partie génération procédurale du terrain, puis
nous regarderons la partie \emph{CDLOD}.\\

La génération procédurale de terrain est très répandue pour obtenir des
surfaces variées et de taille conséquente sans avoir à les créer à la
main. De nombreux logiciels mettent en pratique ces méthodes, par
exemple Terragen\footnote{\url{https://planetside.co.uk/}}, logiciel de
création de mondes en 3D utilisé entre autres dans le monde du cinéma,
ou encore des jeux vidéo comme par exemple Elite:
Dangerous\footnote{\url{https://www.elitedangerous.com/}}. Les
techniques employées dans ces logiciels sont souvent déjà disponibles
implémentées dans des bibliothèques, comme
libnoise\footnote{\url{http://libnoise.sourceforge.net/}} ou
GLM\footnote{\url{https://glm.g-truc.net/0.9.8/index.html}}.\\

Dans son article~\cite{CDLOD}, F. Strugar propose une implémentation
complète de son algorithme, sous licence libre
(zlib\footnote{\url{https://opensource.org/licenses/zlib-license.php}, dernier accès Mars 2018}).
Ce projet propose un affichage de démonstration d'un terrain plat
permettant de voir le fonctionnement de l'algorithme. Cette
implémentation est en C++, cependant elle utilise la bibliothèque
graphique DirectX9, réservée au système d'exploitation Windows et
incompatible avec Linux. Aussi, l'algorithme original étant prévu pour
des terrains plats, l'implémentation n'est pas faite pour les maillages
sphériques. Ce projet n'est pas intéressant à reprendre car il nécessite
beaucoup trop de changements.

Un autre candidat potentiel est le projet WorldGenerator de Leif
Erkenbrach\footnote{\url{http://leifnode.com/} |
\url{https://github.com/LeifNode/World-Generator}, dernier accès: mars 2018} sous licence MIT. Ce
projet permet de visualiser une planète en trois dimensions générée
procéduralement. Implémenté en C++ et utilisant OpenGL 4.4, bibliothèque
graphique disponible pour Linux. Cependant, le projet n'implémente pas
l'algorithme complet de CDLOD, et nécessite d'être porté sous Linux.
Nous avions décidé que ce projet était une base viable, et avons
commencé à le porter sous Linux. Nous nous sommes rendu compte que le
projet ne respectait pas correctement le standard OpenGL 4.5. Corriger
ce problème nécessiterait de réécrire une bonne partie du pipeline
graphique (processus de rendu graphique). Nous avons donc décidé de
chercher une autre base.

Enfin, une troisième base intéressante est le projet Planet Renderer de
Robert
Lindner\footnote{\url{https://github.com/Illation/PlanetRenderer}, dernier accès avril 2018}. Ce
projet, sous licence MIT, affiche une planète en trois dimensions ainsi
que son maillage. L'algorithme de CDLOD est respecté. Le projet est
écrit en C++ et avec OpenGL 4.5. Il est prévu pour Linux et Windows.
Après tests, le projet est une base intéressante, et après concertations
avec le client, la licence lui convient. Nous avons donc décidé de
l'utiliser.
