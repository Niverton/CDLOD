\chapter*{État de l'art}
\addcontentsline{toc}{chapter}{État de l'art}
\setcounter{chapter}{0}

La génération procédurale de terrain est très répandue pour obtenir des
surfaces variées et de taille conséquente sans avoir à les créer à la
main. De nombreux logiciels mettent en pratique ces méthodes, par
exemple Terragen\footnote{\url{https://planetside.co.uk/}}, logiciel de
création de mondes en 3D utilisé entre autres dans le monde du cinéma,
ou encore des jeux vidéo comme par exemple Elite:
Dangerous\footnote{\url{https://www.elitedangerous.com/}}. Les
techniques employées dans ces logiciels sont souvent déjà disponibles
implémentées dans des bibliothèques, comme
libnoise\footnote{\url{http://libnoise.sourceforge.net/}} ou
GLM\footnote{\url{https://glm.g-truc.net/0.9.8/index.html}}.\\

La gènération procédurale de terrains et son rendu en temps interactif pose un problème de complexité algorithmique. En effet les ensembles de données géométriques peuvent être trop complexes pour que l'on puisse les utiliser. Une des solutions possibles est alors de simplifier les détails éloignés de la caméra. Ceci est connu sous le nom de "\emph{Level Of Detail"}(LOD).

\section*{Level of detail}

Le concept de LOD a été introduit en 1976 par James H.Clark dans son article "\emph{Hierarchical geometric models for visible surface algorithms}"~\cite{Clark}. Le but initial de sa recherche était l'amélioration de la qualité d'une structure géométrique ou de l'amélioration de la performance des algorithmes permettant leur génération. L'algorithme présenté par James H.Clark dans son article est l'algorithme de "\emph{Discrete LOD}" (DLOD). La technique utilisée est de créer des LOD pour chaque objet séparément dans un pré-processus. Puis au moment de l'exécution, choisir la LOD de chaque objet en fonction d'un critère comme la distance de l'objet par rapport à la caméra.

Une autre technique utilisée est celle de \emph{Continuous LOD} introduite en 1997 par P.Enrico et S.Roberto dans leurs article "\emph{Simplification LOD and Multiresolution}~\cite{Enrico}. Là où la méthode \emph{Discrete LOD} va créer les LOD dans un pré-processus, la méthode  \emph{Continuous LOD} quand à elle crée une structure de données contenant tous les LOD et va venir les sélectionner lors du rendu. Cette technique va notamment être associée à la méthode de "\emph{View-dependent simplification of
arbitrary polygonal environments}"~\cite{view-dependent} publiée la même année par D.Luebke et C.Erikson. Cette méthode se sert du champ de vision et la distance entre la caméra et l'objet comme critère de choix de LOD.
%pour simplifier ce dernier.\\

\section*{Adaptation de LOD pour le rendu de terrain}

Dans le contexte de la génération de terrain l'utilisation de ces algorithmes permet l'amélioration des performances de rendu.
Les premières méthodes de rendu de terrain appliquant les techniques de LOD utilisaient le CPU, telle que la méthode de "Real-time Optimally Adapting Meshes" (ROAMing terrain) en 1997~\cite{ROAM}.

La méthode de rendu de terrain utilisée lors de ce projet est une méthode qui utilise le GPU, celle de "\emph{Continuous Distance-Dependent Level of Detail for Rendering Heightma}" (CDLOD) par Filip Strugar~\cite{CDLOD}. En effet d'après F.Strugar : "Dans le contexte des PC moderne et des consoles de jeux, il n'y a aucun bénéfice à avoir un algorithme de rendu de terrain qui produit une distribution optimale des triangles sur CPU s'il ne peut produire assez de triangles ou qu'il consomme une trop grosse partie de la puissance de traitement du CPU. Pour que le système puisse être optimal, les algorithmes de rendu de terrains doivent changer pour utiliser les GPU le plus possible". La méthode CDLOD étant une des exigences de notre projet, nous la présentons plus en détail dans le chapitre \ref{chap:cdlod}.

La première méthode de rendu de terrain utilisant le GPU est celle de Ulrich.T~\cite{CLOD} en 2002, sur laquelle F.Strugar se repose notamment dans sa méthode CDLOD (un résumé de la méthode de Ulrich.T est disponible en annexe \ref{sec:chunked-lod}).

\section*{Projets existants appliquant les techniques de LOD}\label{sec:lard-projets}

Dans son article~\cite{CDLOD}, F. Strugar propose une implémentation
complète de son algorithme, sous une licence libre
(zlib\footnote{\url{https://opensource.org/licenses/zlib-license.php},
dernier accès mars 2018}). Ce projet propose un affichage de
démonstration d'un terrain plat permettant de voir le fonctionnement de
l'algorithme. Cette implémentation est en C++, cependant elle utilise la
bibliothèque graphique DirectX9, réservée au système d'exploitation
Windows et incompatible avec Linux. Aussi, l'algorithme original étant
prévu pour des terrains plats, l'implémentation n'est pas faite pour les
maillages sphériques.

Le projet "WorldGenerator" de Leif
Erkenbrach\footnote{\url{http://leifnode.com/} |
\url{https://github.com/LeifNode/World-Generator}, dernier accès: mars
2018} sous une licence MIT, utilise également les techniques de LOD. Ce projet permet de visualiser une planète en
trois dimensions générée procéduralement. Implémenté en C++ et utilisant
OpenGL 4.4, bibliothèque graphique disponible pour Linux. Cependant, le
projet n'implémente pas l'algorithme complet de CDLOD, et dans le cadre de notre projet, nécessite
d'être porté sous Linux. Nous avions décidé que ce projet était une base
viable, et avons commencé à le porter sous Linux. Cependant, nous nous
sommes rendu compte lors de tests que les
shaders\footnote{\texttt{Shader:} programme effectuant une partie du
rendu sur le processeur graphique} utilisés dans le projet
n'étaient pas compatibles avec les drivers graphique Linux.  Corriger ce
problème nécessitait de réécrire une bonne partie du pipeline graphique
(processus de rendu graphique). 

Le projet "Planet Renderer" de Robert
Lindner\footnote{\url{https://github.com/Illation/PlanetRenderer},
dernier accès avril 2018} ,sous licence MIT, affiche une
planète en trois dimensions ainsi que son maillage. L'algorithme de
CDLOD est respecté. Le projet est écrit en C++ et avec OpenGL 4.5 et il
est prévu pour Linux et Windows. Après tests, le projet est une base
intéressante.\\ 

En vu de nos besoins sur l'algorithme à implémenter et sur les
conditions de l'implémentation (voir l'analyse des
besoins~\ref{cahier-des-besoins}), le projet PlanetRenderer est une base
intéressante. Le système d'exploitation (Linux) et la bibliothèque
graphique utilisée (OpenGL) sont adaptés à nos besoins, et l'algorithme
de CDLOD est entièrement implémenté.
